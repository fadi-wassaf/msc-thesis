\chapter{Introduction}

Over the past two decades, superconducting circuits have shown much promise toward the realization of viable quantum computers. Current efforts are dedicated to scaling these devices so that they may one day outperform classical computers in solving complex problems such as simulating quantum systems \cite{Feynman1982} or prime factorization \cite{prime}. 

Enhancing the capabilities of these devices necessitates an increase in the number of qubits used, while simultaneously ensuring that qubit lifetimes and gate fidelities remain high. Additional efforts of scaling superconducting quantum computers have resulted in the construction of qubit lattices, which aim to test quantum simulation methods \cite{evidence_utility,karamlou2023probing,Barends2016} and quantum error correcting codes with the intent to protect against physical errors \cite{google_surface, surface17}. Scaling these devices, however, will ultimately increase their complexity, leading to unexpected parasitic qubit interactions and decay channels that have not previously caused problems. For this reason, it is important to use modeling methods that expose and quantify the potential effects caused by parasitic coupling present in a circuit.

In this thesis, we explore methods for modeling and characterizing superconducting devices that utilize multiport impedance functions. The multiport impedance function contains all of the information about the coupling within a network. Naturally, the characterization of multi-qubit models derived from an impedance function provides us with the coupling rates between all the qubits and resonant modes within a circuit. This will include the effects of any parasitic resonances, stray coupling between qubits, and any contributions to control crosstalk. Using the impedance has the additional benefit that our characterization methods can be applied to electromagnetic simulations, and to circuit models containing both lumped elements (e.g. capacitors and inductors) and distributed elements (e.g. ideal transmission lines).

In Chapter \ref{chapter:lossless_impedance}, we will discuss the lossless reciprocal impedance function and its synthesis. The synthesis then leads to a method that can be used to generally interconnect rational impedance functions. This then allows us to build circuit models for qubit networks given an arbitrary impedance function. Chapter \ref{chapter:analysis_impedance} will explain how to characterize the resulting multi-qubit models by computing effective qubit coupling rates, dispersive shifts in resonant modes and qubit relaxation times. This characterization process is rather simple after obtaining the impedance function. Finally, in Chapter \ref{chapter:examples}, we examine how these methods are used for lumped and distributed element models, in addition to the full electromagnetic simulations. We also showcase a ``brick building” approach for electromagnetic modeling that can help decrease simulation time, thus making the simulation and characterization process more efficient. The models, data, and code implementation used for the examples within Chapter \ref{chapter:examples} are all available at \url{https://github.com/fadi-wassaf/msc-thesis}.
