\chapter{Conclusion and Outlook}

This study explored how to use multiport lossless impedance functions to construct and characterize models of multi-qubit superconducting circuits. In Chapter \ref{chapter:lossless_impedance}, we formulated the CL cascade synthesis of the lossless reciprocal impedance function. We then discussed the analysis of a CL cascade which, when paired with the synthesis, can be used to generally interconnect rational impedance functions. Then, we examined how vector fitting can be used to obtain a rational approximation of a discretized lossless impedance function. This allowed us to construct a lumped circuit model that corresponds to a distributed element model or electromagnetic simulation.

Chapter \ref{chapter:analysis_impedance} shows how to construct a Hamiltonian for a multi-qubit circuit that corresponds to a rational impedance function. We focus on transmon networks, and directly see how the coupling rates between the qubits and resonant modes are related directly to the residues of the impedance function. Then, by approximately block diagonalizing the Hamiltonian, we find an effective qubit Hamiltonian with new shifted qubit frequencies and effective coupling rates that are valid in the dispersive regime. Furthermore, we find expressions for the dispersive shifts in the resonator frequencies that are dependent on the qubit states. In this chapter, we also showed that from a classical circuit perspective, we can compute the relaxation times of all the resonant modes in the circuit due to loss through the external ports of the network.

Finally, in Chapter \ref{chapter:examples}, we present a few examples that showcase the methods discussed in Chapter \ref{chapter:lossless_impedance} and Chapter \ref{chapter:analysis_impedance}. Notably, we show how in a multi-qubit circuit, a lower bound for the qubit decay rates can be found from the diagonal elements of a lossy admittance function where the external ports are shunted by resistors. Additionally, we see how the vector fitting methods can provide us with a circuit Hamiltonian that corresponds to distributed element circuits or electromagnetic models. We also demonstrate how to use the vector fitting and the rational impedance interconnection method alongside a ``brick building” approach to construct larger models out of simpler components.

In the future, it may be worth extending the methods we have investigated to admittance parameters. This may potentially allow for a better handling of circuits that contain a positive semi-definite DC residue (e.g. port with a flux line). The technique of using the circuit Lagrangians may also demonstrate useability in other lossless circuit analysis and synthesis problems. This would allow the interconnection methods to be generalized even further. In obtaining our immittance functions from simulation results, the traditional vector fitting methods we have used do not yield passive or lossless models. For this reason, exploring new vector fitting methods specifically designed for lossless networks may be quite advantageous for the methods we've discussed. 
